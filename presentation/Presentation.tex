%------------------------------------------------------------------
% Präsentation "Suffix Trees und Suffix Arrays"
%------------------------------------------------------------------

% Autoren   : Simon Kaltenbacher, Evgeny Novoseltsev
% Semester  : WS 11/12
% Vorlesung : Effiziente funktionale Datenstrukturen
% Betreuer  : Prof. Martin Hofmann, PhD

% Latex Beamer Tutorial kann unter folgendem Link gefunden werden: http://www.math.umbc.edu/~rouben/beamer/quickstart-Z-H-18.html#node_sec_18

% Eine nützliche Sammlung von Beispielen für die Latex Beamer-Klasse kann unter folgendem Link gefunden werden: http://www.informatik.uni-freiburg.de/~frank/latex-kurs/latex-kurs-3/Latex-Kurs-3.html

\documentclass{beamer}
\usepackage[ngerman]{babel}
% Umlaute in Ausgabedokument anzeigen
\usepackage[T1]{fontenc}
% Kodierung festlegen$
%\usepackage{ucs}
\usepackage[utf8]{inputenc}
% Mathematische Notationen
\usepackage{amsmath, amsfonts, amssymb}
% Bäume zeichnen
\usepackage{tikz}

% Beamer Konfiguration
\usetheme{Goettingen}
%\usecolortheme{sidebartab}
%\useoutertheme{default}
\beamersetuncovermixins{\opaqueness<1>{25}}{\opaqueness<2->{15}}

\usetikzlibrary{trees, calc}

% Styledefinitionen für Suffix Trees
\tikzstyle{branch} = [shape = circle, minimum size = 1mm, inner sep = 0mm, outer sep = 0.5mm, fill]

% Titel
\title{Suffix Trees und Suffix Arrays}  

% Autoren
\author{Simon Kaltenbacher, Evgeny Novoseltsev}

% Erstellungsdatum
\date{\today}

% Dokument
\begin{document}

%------------------------------------------------------------------
% Titelfolie
%------------------------------------------------------------------

\frame{\titlepage}

%------------------------------------------------------------------
% Inhaltsverzeichnis
%------------------------------------------------------------------

\frame{\frametitle{Inhaltsverzeichnis}\tableofcontents}

%------------------------------------------------------------------
% Einführung
%------------------------------------------------------------------

\section{Einführung}
\begin{frame}
\frametitle{Titel}
Merged!
\end{frame}

%------------------------------------------------------------------
% Terminologie
%------------------------------------------------------------------

\section{Terminologie}

%------------------------------------------------------------------
% Suffix Tree Klassen
%------------------------------------------------------------------

\subsection{Suffix Tree Klassen}

\begin{frame}
\frametitle{Atomic Suffix Tree}
$ast$ für die Zeichenkette $agcgacgag$.
\begin{figure}
\begin{tikzpicture}[
    font = \small,
    level 1/.style = {level distance = 7mm, sibling distance = 30mm},
    level 2/.style = {level distance = 7mm, sibling distance = 20mm},
    level 3/.style = {level distance = 7mm, sibling distance = 10mm},
    edge from parent/.style = {->, draw}
]
    \node[branch]{}
    child {
        node[branch]{}
        child {
            node[branch]{}
            child {
                node[branch]{}
                child {
                    node[branch]{}
                    child {
                        edge from parent
                        node[left]{$g$}
                    }
                    edge from parent
                    node[left]{$a$}
                }
                edge from parent
                node[left]{$g$}
            }
            edge from parent
            node[left]{$c$}
        }
        child {
            node[branch]{}
            child {
                node[branch]{}
                child {
                    node[branch]{}
                    child {
                        node[branch]{}
                        child {
                            node[branch]{}
                            child {
                                node[branch]{}
                                    child {
                                        node[branch]{}
                                        child {
                                            edge from parent
                                            node[left]{$g$}
                                        }
                                        edge from parent
                                        node[left]{$a$}
                                    }
                                edge from parent
                                node[left]{$g$}
                            }
                            edge from parent
                            node[left]{$c$}
                        }
                        edge from parent
                        node[left]{$a$}
                    }
                    edge from parent
                    node[left]{$g$}
                }
                edge from parent
                node[left]{$c$}
            }
            edge from parent
            node[right]{$g$}
        }
        edge from parent
        node[above]{$a$}
    }
    child {
        node[branch]{}
        child {
            node[branch]{}
            child {
                node[branch]{}
                child {
                    node[branch]{}
                    child {
                        node[branch]{}
                        child {
                            node[branch]{}
                            child {
                                edge from parent
                                node[left]{$g$}
                            }
                            edge from parent
                            node[left]{$a$}
                        }
                        edge from parent
                        node[left]{$g$}
                    }
                    edge from parent
                    node[left]{$c$}
                }
                child {
                    edge from parent
                    node[right]{$g$}
                }
                edge from parent
                node[left]{$a$}
            }
            edge from parent
            node[left]{$g$}
        }
        edge from parent
        node[left]{$c$}
    }
    child {
        node[branch]{}
        child {
            node[branch]{}
            child {
                node[branch]{}
                child {
                    node[branch]{}
                    child {
                        node[branch]{}
                        child {
                            edge from parent
                            node[left]{$g$}
                        }
                        edge from parent
                        node[left]{$a$}
                    }
                    edge from parent
                    node[left]{$g$}
                }
                edge from parent
                node[left]{$c$}
            }
            child {
                edge from parent
                node[right]{$g$}
            }
            edge from parent
            node[left]{$a$}
        }
        child {
            node[branch]{}
            child {
                node[branch]{}
                child {
                    node[branch]{}
                    child {
                        node[branch]{}
                        child {
                            node[branch]{}
                            child {
                                node[branch]{}
                                child {
                                    edge from parent
                                    node[left]{$g$}
                                }
                                edge from parent
                                node[left]{$a$}
                            }
                            edge from parent
                            node[left]{$g$}
                        }
                        edge from parent
                        node[left]{$c$}
                    }
                    edge from parent
                    node[left]{$a$}
                }
                edge from parent
                node[left]{$g$}
            }
            edge from parent
            node[right]{$c$}
        }
        edge from parent
        node[above]{$g$}
    };
\end{tikzpicture}
\end{figure}
\end{frame}

\begin{frame}
\frametitle{Position Suffix Tree and Compact Suffix Tree}
$pst$ für die Zeichenkette $agcgacgag$.
\begin{figure}
\begin{tikzpicture}[
    font = \small,
    level 1/.style = {level distance = 7mm, sibling distance = 30mm},
    level 2/.style = {level distance = 7mm, sibling distance = 20mm},
    level 3/.style = {level distance = 7mm, sibling distance = 10mm},
    edge from parent/.style = {->, draw}
]
    \node[branch]{}
    child {
        node[branch]{}
        child {
            edge from parent
            node[left]{cgag}
        }
        child {
            edge from parent
            node[right]{gcgacgag}
        }
        edge from parent
        node[above]{a}
    }
    child {
        node[branch]{}
        child {
            node[branch]{}
            child {
                node[branch]{}
                child {
                    edge from parent
                    node[left]{cgag}
                }
                child {
                    edge from parent
                    node[right]{g}
                }
                edge from parent
                node[left]{a}
            }
            edge from parent
            node[left]{g}
        }
        edge from parent
        node[left]{c}
    }
    child {
        node[branch]{}
        child {
            node[branch]{}
            child {
                node[branch]{}
                edge from parent
                node[left]{cgag}
            }
            child {
                node[branch]{}
                edge from parent
                node[right]{g}
            }
            edge from parent
            node[left]{a}
        }
        child {
            edge from parent
            node[right]{cgacgag}
        }
        edge from parent
        node[above]{g}
    };
\end{tikzpicture}
\end{figure}

$cst$ für die Zeichenkette $agcgacgag$.
\begin{figure}
\begin{tikzpicture}[
    font = \small,
    level 1/.style = {level distance = 7mm, sibling distance = 30mm},
    level 2/.style = {level distance = 7mm, sibling distance = 15mm},
    level 3/.style = {level distance = 7mm, sibling distance = 10mm},
    edge from parent/.style = {->, draw}
]
    \node[branch]{}
    child {
        node[branch]{}
        child {
            edge from parent
            node[left]{cgag}
        }
        child {
            edge from parent
            node[right]{gcgacgag}
        }
        edge from parent
        node[above]{a}
    }
    child[level distance = 16mm] {
        node[branch]{}
        child {
            edge from parent
            node[left]{cgag}
        }
        child {
            edge from parent
            node[right]{g}
        }
        edge from parent
        node[left]{cga}
    }
    child {
        node[branch]{}
        child {
            node[branch]{}
            child {
                node[branch]{}
                edge from parent
                node[left]{cgag}
            }
            child {
                node[branch]{}
                edge from parent
                node[right]{g}
            }
            edge from parent
            node[left]{a}
        }
        child {
            edge from parent
            node[right]{cgacgag}
        }
        edge from parent
        node[above]{g}
    };
\end{tikzpicture}
\end{figure}
\end{frame}

%------------------------------------------------------------------
% wotd-Algorithmus: Schritte
%------------------------------------------------------------------

\section{wotd-Algorithmus}

\begin{frame}[t]
\frametitle{wotd-Algorithmus für cst}

Write-only top-down construction for suffix trees.\\

\bigskip

Schritte:
\begin{enumerate}
\item Finde alle Siffixe vom gegebenen Wort.
\item Gruppiere nach den Anfangsbuchstaben.
\item Für jede Gruppe:
\begin{enumerate}
\item Finde das längste gemeinsame Präfix, erzeuge eine Kante mit der Beschriftung, entferne das Präfix von Wörter der Gruppe.
\item Falls die Gruppe leer ist, gehe zu der nächste Gruppe.
\item Falls es in der Gruppe noch Wörter gibt, kehre zum Schritt 2 zurück.
\end{enumerate} 
\end{enumerate}

\end{frame}

%------------------------------------------------------------------
% wotd-Algorithmus: Beispiel
%------------------------------------------------------------------

\subsection{cst-Konstruktion für das Wort $agcgacgag$}

\begin{frame}[t]
\frametitle{Schritt №1 und Schritt №2}

a g c g a c g a g \\
g c g a c g a g \\
c g a c g a g \\
g a c g a g \\
a c g a g \\
c g a g \\
g a g \\
a g \\
g \\
\end{frame}

\end{document}