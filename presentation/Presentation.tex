%------------------------------------------------------------------
% Präsentation "Suffix Trees und Suffix Arrays"
%------------------------------------------------------------------

% Autoren   : Simon Kaltenbacher, Evgeny Novoseltsev
% Semester  : WS 11/12
% Vorlesung : Effiziente funktionale Datenstrukturen
% Betreuer  : Prof. Martin Hofmann, PhD

% Eine nützliche Sammlung von Beispielen für die Latex Beamer-Klasse kann unter folgendem Link gefunden werden: http://www.informatik.uni-freiburg.de/~frank/latex-kurs/latex-kurs-3/Latex-Kurs-3.html

\documentclass{beamer}
\usepackage[ngerman]{babel}
% Umlaute in Ausgabedokument anzeigen
\usepackage[T1]{fontenc}
% Kodierung festlegen$
%\usepackage{ucs}
\usepackage[utf8]{inputenc}
% Mathematische Notationen
\usepackage{amsmath, amsfonts, amssymb}
% Bäume zeichnen
\usepackage{tikz}

% Beamer Konfiguration
\usetheme{Goettingen}
%\usecolortheme{sidebartab}
%\useoutertheme{default}
\beamersetuncovermixins{\opaqueness<1>{25}}{\opaqueness<2->{15}}

% Titel
\title{Suffix Trees und Suffix Arrays}  

% Autoren
\author{Simon Kaltenbacher, Evgeny Novoseltsev}

% Erstellungsdatum
\date{\today}

% Dokument
\begin{document}

%------------------------------------------------------------------
% Titelfolie
%------------------------------------------------------------------

\frame{\titlepage}

%------------------------------------------------------------------
% Inhaltsverzeichnis
%------------------------------------------------------------------

\frame{\frametitle{Inhaltsverzeichnis}\tableofcontents}

%------------------------------------------------------------------
% Einführung
%------------------------------------------------------------------

\section{Einführung}
\begin{frame}
\frametitle{Titel}
Die einzelnen Frames sollten einen Titel haben
\end{frame}

%------------------------------------------------------------------
% Terminologie
%------------------------------------------------------------------

\section{Terminologie}

%------------------------------------------------------------------
% Suffix Tree Klassen
%------------------------------------------------------------------

\subsection{Suffix Tree Klassen}

\begin{frame}
\frametitle{Atomic Suffix Tree}
\end{frame}

\begin{frame}
\frametitle{Position Suffix Tree}
\end{frame}

\begin{frame}
\frametitle{Compact Suffix Tree}
\end{frame}

\end{document}