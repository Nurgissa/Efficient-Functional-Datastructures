%////////////////////////////////////////////////////////////////////////////
% Präsentation "Suffix Trees und Suffix Arrays"
%////////////////////////////////////////////////////////////////////////////

% Autoren   : Simon Kaltenbacher, Evgeny Novoseltsev
% Semester  : WS 11/12
% Vorlesung : Effiziente funktionale Datenstrukturen
% Betreuer  : Prof. Martin Hofmann, PhD

%////////////////////////////////////////////////////////////////////////////

\documentclass[12pt]{article}

%////////////////////////////////////////////////////////////////////////////
% Import
%////////////////////////////////////////////////////////////////////////////

\usepackage{graphicx}    % needed for including graphics e.g. EPS, PS
\usepackage[ngerman]{babel}
\usepackage{textcomp}
% Umlaute in Ausgabedokument anzeigen
\usepackage[T1]{fontenc}
% Kodierung festlegen
%\usepackage{ucs}
\usepackage[utf8]{inputenc}
% Mathematische Notationen
\usepackage{amsmath, amsthm, amsfonts, amssymb, mathtools, xfrac}
% Code von Programmiersprachen erkennen
% \usepackage{courier}
\usepackage{inconsolata}
\usepackage{listings, xcolor}
% Bäume zeichnen
\usepackage{tikz}
% Tabellen
\usepackage{booktabs}
\usepackage{tabularx}
% \usepackage{ctable}
% Literaturverzeichnis
\usepackage[numbers, square]{natbib}

%////////////////////////////////////////////////////////////////////////////
% Konfiguration
%////////////////////////////////////////////////////////////////////////////

%--Article Konfiguration------------------------------------------------------

\topmargin -1.5cm        % read Lamport p.163
\oddsidemargin -0.04cm   % read Lamport p.163
\evensidemargin -0.04cm  % same as oddsidemargin but for left-hand pages
\textwidth 16.59cm
\textheight 21.94cm 
%\pagestyle{empty}       % Uncomment if don't want page numbers
\parskip 7.2pt           % sets spacing between paragraphs
%\renewcommand{\baselinestretch}{1.5} 	% Uncomment for 1.5 spacing between lines
\parindent 0pt		  	 % sets leading space for paragraphs

%--colors--------------------------------------------------------------------

\definecolor{verylightgray}{RGB}{230, 230, 230}
\definecolor{rubineRed}{RGB}{243, 0, 125}
\definecolor{aquamarine}{RGB}{0, 193, 223}

%--listings-Konfiguration----------------------------------------------------

\definecolor{verylightgray}{RGB}{230, 230, 230}
\lstloadlanguages{Haskell}
\lstset {
    language = Haskell,
    tabsize = 4,
    showstringspaces = false,
    breaklines = true,
    basicstyle = \ttfamily\fontsize{8}{8}\selectfont,
    keywordstyle = \color{blue},
    commentstyle = \color{darkgray},
    frame = lines,
    extendedchars = true,
    framextopmargin = 5pt,
    xleftmargin = 5pt,
    framexleftmargin = 5pt,
    framexrightmargin = 5pt,
    framexbottommargin = 5pt,
    framerule = 0.5pt,
    rulecolor = \color{gray},
    backgroundcolor = \color{verylightgray}
}

%--TikZ Konfiguration--------------------------------------------------------

\usetikzlibrary{trees, calc, shapes, scopes}

%--Styledefinitionen für Suffix Trees----------------------------------------

\tikzstyle{branch} = [
    shape = circle,
    minimum size = 1mm,
    inner sep = 0mm,
    outer sep = 0.5mm,
    fill
]
\tikzstyle{ghostbranch} = [
    shape = circle,
    draw,
    minimum size = 1mm,
    inner sep = 0mm,
    outer sep = 0.5mm
]
\tikzstyle{leaf} = [
    inner sep = 0mm,
    outer sep = 0mm
]
\tikzstyle{subtree} = [
    isosceles triangle,
    dotted,
    draw,
    shape border rotate = 90,
    isosceles triangle stretches = true,
    minimum height = 10mm,
    minimum width = 10mm,
    inner sep = 0,
    anchor = north,
    font = \tiny
]

%--Literaturverzeichnis------------------------------------------------------

% Literaturverzeichnis Stil
\bibliographystyle{alpha}

%--Benutzerdefinierte Makros-------------------------------------------------

% Umgebung Bemerkung
\newtheorem{remark}{Bemerkung}

\newcommand{\coloneqq}{\mathrel{\mathop:}=}

\newcommand{\abs}[1]{\left|#1\right|}

% Hack
\newcommand{\newblock}{}

%////////////////////////////////////////////////////////////////////////////
% Dokumenteigenschaften
%////////////////////////////////////////////////////////////////////////////

% Titel
\title{Suffix Trees und Suffix Arrays}

% Autoren
\author{Simon Kaltenbacher, Evgeny Novoseltsev}

% Erstellungsdatum
\date{\today}

%////////////////////////////////////////////////////////////////////////////
% Dokument
%////////////////////////////////////////////////////////////////////////////

\begin{document}     

%--Seite: Titel--------------------------------------------------------------
\maketitle    
\newpage

%--Seite: Inhaltsverzeichnis-------------------------------------------------
\tableofcontents
\newpage

%////////////////////////////////////////////////////////////////////////////
\section{Einführung}

% Für Referenzen zu den Labels benutze \ref{name}
\label{Einführung} 
%////////////////////////////////////////////////////////////////////////////

%////////////////////////////////////////////////////////////////////////////
\section{Suffix Tree}
\label{Suffix Tree}
%////////////////////////////////////////////////////////////////////////////

%////////////////////////////////////////////////////////////////////////////
\subsection{Terminologie}
\label{Terminologie}
%////////////////////////////////////////////////////////////////////////////

%////////////////////////////////////////////////////////////////////////////
\subsubsection{Suffix Tree Klassen}
\label{Suffix Tree Klassen}
%////////////////////////////////////////////////////////////////////////////

%////////////////////////////////////////////////////////////////////////////
\subsubsection{Generalized Suffix Tree}
\label{Generalized Suffix Tree}
%////////////////////////////////////////////////////////////////////////////

%////////////////////////////////////////////////////////////////////////////
\subsection{Anwendungsgebiete}
\label{Anwendungsgebiete}
%////////////////////////////////////////////////////////////////////////////

%////////////////////////////////////////////////////////////////////////////
\subsubsection{Überblick}
\label{Überblick}
%////////////////////////////////////////////////////////////////////////////

%////////////////////////////////////////////////////////////////////////////
\subsubsection{Longest common substring}
\label{Longest common substring}
%////////////////////////////////////////////////////////////////////////////

%////////////////////////////////////////////////////////////////////////////
\subsection{Suffix Tree Konstruktion}
\label{Suffix Tree Konstruktion}
%////////////////////////////////////////////////////////////////////////////

%////////////////////////////////////////////////////////////////////////////
\subsubsection{WOTD-Algorithmus}
\label{WOTD-Algorithmus}
%////////////////////////////////////////////////////////////////////////////

%////////////////////////////////////////////////////////////////////////////
\subsubsection{Laufzeitanalyse}
\label{Laufzeitanalyse}
%////////////////////////////////////////////////////////////////////////////

%////////////////////////////////////////////////////////////////////////////
\subsection{Teilwortsuche}
\label{Teilwortsuche}
%////////////////////////////////////////////////////////////////////////////

%////////////////////////////////////////////////////////////////////////////
\section{Suffix Array}
\label{Suffix Array}
%////////////////////////////////////////////////////////////////////////////

%////////////////////////////////////////////////////////////////////////////
\subsection{Motivation}
\label{Motivation}
%////////////////////////////////////////////////////////////////////////////

%////////////////////////////////////////////////////////////////////////////
\subsection{DC3-Algorithmus}
\label{DC3-Algorithmus}
%////////////////////////////////////////////////////////////////////////////


\end{document}